\subsection{Ownership and so on…}

From the linear types paper:

\begin{spec}
  newMArray    :: (MArray a ⊸ Unrestricted b) ⊸ Unrestricted b
  writeMArray  :: MArray a ⊸ Int -> a -> MArray a
  freeze       :: MArray a ⊸ Unrestricted (Array a)
  readArray    :: Array a -> Int -> a
\end{spec}

In |writeArray|: we insert an unrestricted element |a ->|. Otherwise
we could do a linear type taboo:

\begin{spec}
unrestrict :: a ⊸ Unrestricted a
unrestrict x = case unrestrictArray x of
  Unrestricted arr -> readArray 0

unrestrictArray :: a ⊸ Unrestricted (Array a)
unrestrictArray x = newArray $ \marr ->
  freeze $
  writeMArray 0 x marr
\end{spec} % emacs <- syntax highlighting bug

This is not ok if I want to make a multidimensional array (say an
|MArray (MArray a)|), which I would later freeze.

How could freeze look like for that use-case?

Something like

\begin{spec}
  freeze :: MArray a ⊸ (a ⊸ Unrestricted b) -> Unrestricted (Array b)
\end{spec}

But this is no longer $O(1)$ unless the compiler has special support
(like |Coercible|~\cite{safe-coercions}).

The crux of the issue is that mutable arrays and immutable arrays have
distinct types, which me must convert.

Contrast with Rust, where there is a single type \verb+Vector+, and
freezing is simply
\verb+fn (vect : Vector<a>) : Rc<Vector<a>> -> { Rc<vect> }+ (check
syntax).

With linear constraints:

\begin{spec}
-- Each reference has 3 linear capabilities associated with it. References can be freely copied, but the capabilities are controlled linearly.

-- The relation between references and capabilities is mediated with an existential type variable (of kind Token for legibility here, but we can let Token = Type with no loss of expressivity.)

kind Token
class Read (n :: Token)    -- read capability
class Write (n :: Token)   -- write capability
class Own (n :: Token)     -- move, free,

type O n = (Read n, Write n, Own n) -- but we cannot move unless no one has kept a reference, so really all 3 capabilities are needed. We never use Own alone, always O.

type RW n = (Read n, Write n) -- likewise, one cannot be writing unless we also have read access (so that a reader does not see changes happening while it is reading)

type Reference = Token -> Type -- kind of types which are associated to capabilities via tokens (so ``references'')

AtomRef a :: Reference

data AtomRef (a :: Type) (n :: Token)

writeRef :: (RW n) =>. AtomRef a n -> a -> () .<= RW n
readRef :: (Read n) =>. AtomRef a n -> a .<= Read n
dealloc :: O n =>. AtomRef a n -> ()
newRef :: (forall n. O n =>. AtomRef a n -> Unrestricted b) ⊸ Unrestricted b
aliasRef :: forall n. (RW n, O p) =>. AtomRef a n -> AtomRef a p -> () .<= RW n

-- Polymorphic aliasing
aliasRef' :: forall n. (RW n, O p, KnownRef a) =>. a n -> a p -> () .<= RW n


data PArray (a :: Reference) (n :: Token)
  -- This an array of References: is an array of boxed stuff. Are all
  -- references inside it are associated with the same token.

PArray a :: Reference

-- type Array a = Frozen (PArray a)
-- data Frozen a where
--   Freezed :: Read n => a n -> Frozen a
-- data Reading a where
--   Read :: Read n =>. a n -> Reading a
-- data Borrowed a where
--   Borrow :: RW n =>. a n -> Borrowed a
-- data Owned (a :: * -> *) where
--   Move :: O n =>. a n -> Owned a

newPArray    :: (forall n. O n =>. PArray a n -> Unrestricted b) ⊸ Unrestricted b
-- Create a new array with the 3 capabilities available (|O n|).

-- Wrong:
-- writePArray  :: (RW n, O p) =>. PArray a n -> Int -> a p -> Owned a .<= RW n
  -- Relinquishes the ownership of the |a p| argument. Returns
  -- ownership of the old value.
  -- iiuc: The reference p is moved inside the array (if we'd make a copy we don't need the O p capability). So we make a new token for it (existentially quantified by Owned).
  -- ? With this interface it seems that the owner will be able to freeze the array, but the reference will survive. So it seems that there is a bug here.

writePArrayAlt  :: (RW n, O p) =>. PArray a n -> Int -> a p -> () .<= (RW n)
-- here the ownership of |a p| is absorbed by n.
-- Operational semantics: move the p *reference* inside n.

readPArray   :: Read n =>. PArray a n -> Int -> a n .<= Read n
-- OK

borrowPArrayElement     :: RW n =>. Parray a n -> Int -> (Borrowed a             ⊸ Unrestricted b         ) ⊸ Unrestricted b .<= RW n
borrowPArrayElementAlt  :: RW n =>. Parray a n -> Int -> (forall p. RW p =>. a p ⊸ Unrestricted b .<= RW p) ⊸ Unrestricted b .<= RW n
borrowPArrayElementAlt' :: RW n =>. Parray a n -> Int -> (forall p. RW p =>. a p ⊸ k              .<= RW p) ⊸              k .<= RW n -- continuation won't be called from an unrestricted context because the initial RW n constraint is unique and linear.

borrowPArrayElement is the only way to access an element in read-write

-- freeze       :: O n =>. PArray a n -> Array a
freeze       :: O n =>. PArray a n -> Array a
-- OK


readArray :: Array a -> Int -> Frozen a
readArray (Freezed arr) i = Freezed (readPArray arr i)
\end{spec}
